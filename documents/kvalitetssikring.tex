\section{Kvalitetssikring}
Vi vil kvalitetssikre vores produkt ved at teste alt det der giver mening at teste,
så vi sikrer at funktionaliteten fungerer efter hensigten. Derudover laver vi reviews på hinandens kode inden
det bliver godkendt og merges til master branchen i git.
For at sikre læsbarhed og konsistens i koden, følger vi nogle specifikke PHP kode standarder som er beskrevet i afsnittet omkring kode standarder.
Efter hvert sprint holder vi et review møde hvor interessenter fra Ordbogen A/S vil deltage og komme med feedback.
Så vi på den måde hele tiden sikrer os at det vi laver, er det Ordbogen gerne vil have.
For at sikre kvaliteten i projektet yderligere vil vi bruge FURPS som kvalitetsmodel og hele tiden
veje projektet op imod FURPS.
