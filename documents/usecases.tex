\section{Use cases}
\subsection{Use case diagram}
Vi har lavet et \hyperref[fig:usecase-complete]{samlet use case diagram}, der viser vores aktører og deres use cases i systemet.\\
For at gøre det mere overskueligt har vi lavet det således at der er indbygget et workflow for hver aktion i systemet.
På denne måde har vi kunnet fremvise dette diagram for vores interessenter, og få feedback på funktionerne i systemet på en samlet måde.
I stedet for at fremvise et diagram for hver enkelt use case, giver det \hyperref[fig:usecase-complete]{samlede use case diagram} også et syn på hvordan de forskellige funktioner snakker sammen.
\begin{figure}[H]
    \makebox[\textwidth]{\includegraphics[scale=0.55]{UseCaseDiagram_complete}}
    \caption{Samlet use case diagram}
    \label{fig:usecase-complete}
\end{figure}
\subsection{Use cases der er færdige}
Da vi ikke har nået at implementere alle use cases, og fordi der til vores sidste review blev bestemt at nogle skulle skrottes,
har vi tilføjet nedenstående \hyperref[fig:usecase-done]{billede}.
\\
De use cases der er markeret med grøn, er dem der er implementeret, og dem med et kryds er dem der er fjernet fra det oprindelige diagram.
(Validér API endpoint er ikke komplet implementeret, men kun delvist, og derfor den gulig-grønne farve)
\begin{figure}[H]
    \makebox[\textwidth]{\includegraphics[scale=0.55]{UseCaseDiagram_done}}
    \caption{Use cases der er færdige}
    \label{fig:usecase-done}
\end{figure}
\subsection{Use case - Gem API endpoint}
\textbf{Aktør:} Udvikler
\\\\
\textbf{Scenarie:} En udvikler vil gemme et API endpoint på vores server da det skal være tilgængeligt i vores system. \\
Brugeren indtaster informationer om sit API endpoint i den givne formular. (Firma, typen af data, url til endpoint). \\
Brugeren modtager herefter en bekræftelse på om data er blevet gemt korrekt.
\\\\
\textbf{Alternative scenarier:} Udvikleren finder det besværligt og får en administrator til at tilføje sit API endpoint,
eller vælger at forlade siden uden at gemme.
\subsection{Systemsekvensdiagram}
Systemsekvensdiagrammet visualiserer hvordan \textit{Gem Api} fungerer i praksis, og hvordan de forskellige lag i vores applikation fungerer.
\\
Vi har valgt at dele det op i frontend og backend, der gør det nemmere at overskue det konkrete opdeling af de to ting.
\\\\
Gem Api går i alt sin enkelhed ud på at brugeren indtaster noget data og forsøger at gemme det i databasen.
Hvis dette ikke lykkedes, får brugeren en fejlbesked. Nedenfor er denne sekvens visualiseret.
\begin{figure}[H]
    \makebox[\textwidth]{\includegraphics[scale=0.50]{Systemsekvensdiagram}}
    \caption{Systemsekvensdiagram for Gem Api}
\label{fig:ssd}
\end{figure}
