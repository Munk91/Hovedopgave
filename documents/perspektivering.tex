\section{Perspektivering}
Projektet er en gennemgåede beskrivelse af vores system, tankerne bag systemet, vores arbejdsproces og de teknikker vi har anvendt.
\\
Kravene til systemet er i hovedtræk opfyldt. \textit{Men hvad skal der ske med systemet udover dette projekt?}
\\\\
Vi fik til vores sidste review en del feedback på, hvad der skulle laves på systemet hvis det skulle være anvendeligt for Ordbogen A/S.
\\
De ønskede et mere intuitivt interface, eventuelt nogle guides til hvordan man uploader endpoints og hvordan man laver diagrammer ud fra statistik data.
Herudover var det vigtigt for dem, at det var nemt at bruge for medarbejdere uden så meget IT forståelse.
\\\\
Noget som Ordbogen A/S syntes var et interessant perspektiv var, at lave et så dynamisk system. Det var ligegyldigt hvem der uploadede endpoints med data til systemet,
så længe det var valid data. Dette ville nemlig give mulighed for fremtidig udvidelse, således at alle virksomheder med relevant statistik data har mulighed for, at gøre dette tilgængeligt.
\\
Dette ville give Ordbogen A/S nogle muligheder i form af, at kunne sammenligne deres statistik med andre virksomheder, eller til bedre at kunne analysere deres kunder og produkter.
\\\\
Selvom der findes mange andre systemer der besidder samme eller lignende funktionalitet som vores, føler vi at systemet er brugbart,
da det har været i Ordbogen A/S' ånd at skabe noget på egen hånd, og tilpasse systemet fuldstændig til virksomhedens egne behov.
Idet de er en IT virksomhed, udnytter de i høj grad, at de kan udvikle deres egne systemer, så de kan laves og tilpasses i deres eget tempo.
Selvom det giver en masse vedligeholdelse og udvikling, skaber det også et godt workflow i deres forskellige afdelinger, da det er nemt for dem at rette deres arbejdsredskaber til.
\\
Vores system ville gå ind under den kategori, hvor de har mulighed for selv at tilpasse alle funktionerne, designet og flowet.
