\section{Refleksion}
\subsection{Systemet som en helhed}
Da vi begge to havde arbejdet med webudvikling i vores praktikforløb, kom mange af de ting vi skulle lave naturligt til os.
\\\\
Selvom vi havde denne erfaring, voldte det dog en del problemer at sætte et system op. Vi kom fra to forskellige teams på Ordbogen A/S, som gjorde det på hver sin måde. 
Derfor var der en del snak frem og tilbage om hvorvidt det skulle gøres. Det endte ud med en god blanding fra begge. 
Under hele forløbet har vi været gode til at tage hensyn til hinandens holdninger og vi har skabt en god debat omkring løsninger. Det har lært os en masse angående samarbejde og kompromiser. 
Generelt har vi kunne trække en masse viden blandt andet konflikthåndtering, som vi blev undervist i på første år. 
\\\\
Selve systemet har givet os en god erfaring med JavaScript, PHP og search engines. 
Selve JavaScript og PHP teknologierne, var lagt ud på forhånden af Ordbogen A/S,
men vi lærte en masse om, hvilke frameworks der eksisterer og hvordan de bruges.
Derudover gav det en masse ny viden indenfor search engines, da ingen af os havde arbejdet med det forinden. 
\\\\
Den teorietiske del af studiet, har hjulpet os igennem forløbet. Det at bevare overblikket og at kunne planlægge opgaver, 
hvor vi hele tiden har taget hensyn til interresenternes krav.
\\
Noget vi også har lært på studiet er, at tilegne os ny viden. Det har været rigtig godt når
der har været en del nye frameworks, arbejdsgange og teknologier i det her projekt.
Vi har lavet et projekt, som vi kan stå inde for og hvor vi har tilegnet os en masse ny viden, som vi kan bruge fremover.
\subsection{ElasticSearch}
Da vi udelukkende har valgt at hente ren statistik data, kunne man have brugt MySQL og vedholdt den høje performance til at hente data ud. 
Valget med at bruge ElasticSearch til at indexere dataen gør, at hvis man i fremtiden vælger at hente rå data ind og lade applikationen håndtere
udvælgelsen af den statistike data, vil performance ikke blive påvirket i samme grad, som med en MySQL database.

Arbejdet med ElasticSeach har givet os en god indsigt i, hvad det vil sige at arbejde med søgeoptimering.
Vi havde en del problemer med hvordan dataen skulle komme ind og hvordan den kom ud igen. 
Da ElasticSearch stadig er en ny spiller på markedet af search engines synes vi, at der er steder hvor dokumentation har haltet.
Det har gjort, at vi har været nødsaget til at være enormt eksperimenterende i forhold til hvordan vi skulle få indekseret data og hvad der blev returneret.
På grund af overstående har vi fået en skarpere forståelse af ElasticSearch, da det har tvunget os til at lave en masse prototyper.
