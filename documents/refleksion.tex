\section{Refleksion}
\subsection{ElasticSearch}
Da vi udelukkende har valgt at hente ren statistik data, kunne man have brugt MySQL og vedholdt den høje performance til at hente data ud. 
Valget med at bruge ElasticSearch til at indexere dataen, gør at hvis man i fremtiden vælger at hente rå data ind og lade applikationen håndtere
udvælgelsen af den statistike data, vil performance ikke blive påvirket i samme grad, som med en MySQL database.

Arbejdet med ElasticSeach har givet os en god indsigt i, hvad det vil sige at arbejde med søgeoptimering.
Vi havde en del problemer med hvordan dataen skulle komme ind og hvordan den kom ud igen. 
Da ElasticSearch stadig er en ny spiller på markedet af search engines synes vi, at der er steder hvor dokumentation har haltet.
Det har gjort, at vi har været nødsaget til at være enormt eksperimenterende i forhold til hvordan vi skulle få indekseret data og hvad der blev returneret.
På grund af overstående har vi fået en skarpere forståelse af ElasticSearch, da det har tvunget os til at lave en masse prototyper.
