\section{Testing}
Her vil vi uddybe de test framework vi har brugt til frontend og backend og komme
med eksempler. I vores eksempler mocker vi objekter der simulerer de rigtige objekter i vores controller.
\subsection{Hvad er testing}
Unit testing er en måde at splitte dit program op i individuelle dele og teste de funktionaliteter.
En unit test er bygget på omkring de tre A'er (arrange, act, assert). Det går ud på at alle
preconditions og input er mødt, herefter bruger man en metode eller et objekt under testen og til sidst at det forventede sker.\footnote{http://c2.com/cgi/wiki?ArrangeActAssert}
\subsection{Jasmine}
Jasmine er et test framework til Javascript. Det kommer med en masse funktioner, 
der gør det overskueligt og nemt at begynde at teste.
Noget af det der bliver lagt vægt på i Jasmine er dokumentation og struktur. Jasmine frameworket er 
også behaviour driven, som udskiller sig fra andre test framework, her beskriver man den forventede opførsel først.
Et eksempel på en test, her fra vores StatisticsController.spec.js fil
\begin{lstlisting}[caption={StatisticsController.spec.js}, language={JavaScript}, label={lst:StatisticsController}]
it('should get the types based on the selected statistic index', function() {
    spyOn(ctrl, 'getStatsTypes').and.callThrough();

    statistic = { type : ['someType'] }

    ctrl.getStatsTypes(statistic);

    expect(ctrl.getStatsTypes).toHaveBeenCalledWith(statistic);
    expect(setActiveDropdownValue).toHaveBeenCalled;
    expect(ctrl.activeStatisticTypeList[0]).toBe('someType');
});
\end{lstlisting}
i kode eksempel~\ref{lst:StatisticsController} starter vi med en beskrivelse af hvad testen gør, her går den ind og og kalder getStatsTypes i vores controller
med noget dummy data også forventer vi at vores activeStatisticTypeList[0] bliver sat med den type fra vores dummy data.

\subsection{PHPUnit}
PHPUnit er det framework vi bruger til vores backend test. ligesom med Jasmine indeholder det
en masse indbyggede funktioner der gør det nemt at gå til.
\begin{lstlisting}[caption={UserControllerTest.php}, language={PHP}, label={lst:UserControllerTest}]
public function testCreateUser() {
    $userMock = $this->getMockBuilder('User')
        ->getMock();

    $userControllerMock = $this->getMockBuilder('UserController')
        ->setMethods(array('createUser'))
        ->getMock();

    $userControllerMock->expects($this->once())
        ->method('createUser')
        ->with($userMock)
        ->will($this->returnValue(true));

    $this->assertTrue($userControllerMock->createUser($userMock));
}
\end{lstlisting}
Som man kan se i kode eksempel~\ref{lst:UserControllerTest} gør PHPUnit det nemt at mocke objekter.
I denne test mocker vi User objektet fra vores Model, vores UserController og til sidst kalder vi userControlleren 
med vores userMock og tester på at brugeren bliver lavet.

