\section{Reflektioner}
\subsection{ElasticSearch}
Når vi har valgt at hente data ud på den måde vi gør nu, efter snak med Ordbogen A/S, ved at der kun kommer statistik data ind fra API'erne
kunne vi nok godt have gjort det ligeså godt ved f.eks kun at bruge MySQL.
Men vores valg med ElasticSearch gør også at vi fremtidsikrer produktet ved ikke at udelukke måden 
at vi henter en masse rå data fra apis og vha. ElasticSearch hurtigt sammensætter dem til statistic data.
Hvis det var at man på et senere tidspunkt ville skifte database, til noget mere enkelt har vores lagdeling også gjort det ret nemt. 
Da kun vores repository snakker sammen med databasen.
Når man videre reflekterer over ElasticSearch har vi gjort en masse fremskridt i forhold til arbejde med søge optimering og
det har givet en god forståelse af hvad det vil sige at opretholde en søgemaskine og de muligheder det giver en med sammmenfletning af data.

