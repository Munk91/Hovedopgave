\section{Projektstyring}
\subsection{Scrum}
Vi har anvendt \textit{Scrum} som værktøj til projektstyring igennem hele forløbet med hovedopgave. Det giver nogle gode retningslinjer for hvordan man skal planlægge et projekt, og giver samtidig mulighed for nemmere at forhindre at ens projekt går i den forkerte retning. I forlængelse af at man typisk vil prøve at undgå fejl eller dårligere perioder samt dårlig arbejdsgang, giver \textit{Scrum} også mulighed for at rette op på de ting der går skidt eller hvis man er utilfreds med noget.

\subsection{Planlægning}
Vores projekt er delt op i nogle perioder, som kaldes for et \textit{sprint}. Et sprint varer for vores vedkommende 2 uger, og består af nogle forskellige faser.
Det starter med \textit{sprint planning}, som er den fase hvor vi planlægger hvad vi skal lave af opgaver inden for sprintet. Vores overordnede opgaver for projektet er beskrevet i en \textit{backlog}, som er defineret ud fra use cases. Fra \textit{backlog'en} trækker vi de opgaver ind i det kommende \textit{sprint}, som vi mener er mulige at lave færdige inden tidsperioden på to uger er gået. Efter de overordnede opgaver er trukket ind i \textit{sprintet}, skal de deles op i mindre og mere håndtèrbare opgaver. Når dette er gjort og vi er enige om hvad opgaven indeholder, kaster vi nogle point på de opdelte opgaver som definerer hvor lang tid vi tror det tager at løse det givne problem. Så snart vi er enige om hvor tidskrævende alle opgaver er, kan \textit{sprintets} største fase begynde hvor vi løser opgaver.
\\\\
Hver dag starter med et \textit{daily stand-up} møde, hvor hver medlem af gruppen fortæller om hvad personen lavede dagen før, hvad der skal laves den pågældende dag og eventuelt hvilke udfordringer der har været eller hvis man har noget vigtigt der skal drøftes.
\\\\
Når et \textit{sprint} er overstået holdes der \textit{sprint review}, hvor vi inviterer interessenter til et møde der går ud på at fremvise hvad vi har lavet i løbet af det forgange \textit{sprint}. På mødet er det også muligt for andre at komme med feedback til produktet eller arbejdsgangen.
\\\\
Til \textit{sprint retrospective} reflekterer vi blandt andet over \textit{sprintets} forløb og konkluderer på feedback. Herefter nedskriver hvert medlem af gruppen et tal fra 1-5, hvor 1 er lavest og 5 er højest. Dette tal summerer op hvor godt man personligt har haft det i løbet af sprintet. Til tallet følger der nogle positive og negative kommentarer, som er en forklaring på det tal hver person er endt med. Vi taler herefter frit fra leveren om de gode og dårlige oplevelser i løbet af sprintet, og så sørger gruppens \textit{scrum master} for at eventuelle bliver løst.
\\\\
Vi har tilføjet en graf der viser det gennemsnitlige humør-tal uge for uge. Den afspejler hvor godt gruppen synes forløbet er gået.
\\\\
TODO: Indsæt happiness graf (sprint 1: 4, sprint 2: )
