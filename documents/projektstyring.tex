\section{Projektstyring}
\subsection{Scrum}
Vi har anvendt \textit{Scrum} som værktøj til projektstyring igennem hele forløbet med hovedopgaven.
Det gav nogle gode retningslinjer for hvordan vi skulle planlægge projektet, samtidig med at det har givet mulighed for nemmere at forhindre at projektet gik i den forkerte retning.
I forlængelse af, at vi typisk har villet prøve at undgå fejl eller dårlige perioder samt dårlig arbejdsgang, gav Scrum os mulighed for
at rette op på de ting der gik skidt eller vi var utilfredse med.

\subsection{Planlægning}
Hvert sprint har varet to uger, hvoraf den første dag var sat af til planlægning af opgaver og prioritering af disse.
Den sidste dag i sprintet har lige så været afsat, til at udarbejde review og retrospective.
\\
Vi har på den måde fået samlet op på sprintet og snakket om hvordan de sidste par uger er gået, både på godt og ondt.
\\\\
Ved at bruge Scrum, har vi hele tiden kunne holde os på rette spor, og lavet de mest relevante opgaver.
Der var f.eks. efter første sprint, et behov for, at ændre i vores opgaver, da vi havde brugt vores tid på irrelevante funktioner i systemet.
Ved at få feedback, kunne vi ændre retning og begynde på noget der havde mere relevans for systemet og vores interessenter.
\\\\
Vi har også haft mulighed for, at ændre på mere personlige ting.
Hvis der har været nogle udfordringer der krævede opmærksomhed, kunne dette også rettes op.
\\
Det kunne eksempelvis forekomme, at vi havde en stresset periode på hjemmefronten, og derfor ikke ville arbejde så fokuseret i en periode.
Denne arbejdsmetode gav os mulighed for, at justere opgavernes omfang, alt efter hvor meget eller hvor lidt vi kunne arbejde.
\\\\
Ved hvert retrospective har vi hver især skrevet et tal fra 1-5, hvor 1 var lavest og 5 var højest, i forhold til hvor godt vi personligt havde haft det i løbet af sprintet.
Til tallet fulgte det nogle positive og negative kommentarer, som var en forklaring på det givne tal hver person var endt på.
Vi talte herefter frit fra leveren om de gode og dårlige oplevelser der havde været i løbet af sprintet, derefter sørgede scrum masteren for at
eventuelt dårlige oplevelser blev løst.
\\\\
Vi har tilføjet en \hyperref[fig:happi]{graf} der viser det gennemsnitlige humør-tal uge for uge.
\\
Den afspejler hvor godt gruppen synes forløbet er gået.
\begin{figure}[H]
    \makebox{\includegraphics[scale=0.75]{Happiness_graf}}
    \caption{Graf der viser vores happiness index i vores sprints}
    \label{fig:happi}
\end{figure}
\noindent
Forklaringen på grafen er, at vi var meget positive efter første sprint, hvor vi havde lavet analyse og forberedelser.
Vi skulle til at kode systemets første funktioner og klasser og var derfor meget fortrøstningsfulde.
\\
Dog gik det ikke helt som forventet, da vi fik negativ feedback og blev nødt til at ændre vores planlægning og prioritering,
og vi blev nærmest nødt til at starte forfra, hvilket også ses midt på grafen.
\\
Vi endte dog med et fornuftigt produkt, og overholdt den sidste deadline. Efter sidste sprint fik vi en del konstruktiv feedback, 
som kan blive afgørende for produktets videreudvikling. Vi har også kunne bruge noget af det til personlig udvikling.

\subsection{Scrum værktøjet YouTrack}
I projektperioden har vi anvendt YouTrack til at visualisere metoder fra Scrum.
\\
Selvom vi har anvendt andre værktøjer på studiet og i praktikken, faldt valget på YouTrack, da dette værktøj er gratis og
har givet os nødvendig funktionalitet, for at kunne planlægge opgaver og holde styr på deres stadier.
\\\\
YouTrack gav os mulighed for at lave rapporter, dog har vi kun har brugt muligheden for at lave et burndown chart. Herudover kunne vi oprette
en backlog og et sprint. Herefter havde vi mulighed for at opdele vores backlog opgaver og trække dem ud i en såkaldt \textit{swim lane}.
Swim lane, har forskellige stadier, \textit{Open}, \textit{In Progress}, \textit{Fixed}, \textit{Verified}. På denne måde kunne vi hele tiden opdateres om, hvor langt i processen vi var nået med de enkelte opgaver, derved blev der givet et overblik over hvad medlemmerne i gruppen lavede eller hvad der manglede i det nuværende sprint.
\\
En mangel i Youtrack har været, at vi ikke har haft mulighed for, at lave en projekt backlog. Denne bruges typisk, til at indeholde de overordnede opgaver i projektet som mangler at blive lavet.
Da vi følte det var nødvendigt at have en sådan backlog, valgte vi at lave et dokument på Google Docs, og gjorde det tilgængeligt for begge medlemmer af gruppen.
Dette dokument brugte vi til at skabe et overblik over hvilke overordnede opgaver vi havde tilbage i projektet, samt tilføje nye opgaver der kunne opstå.
Listen var delt op i opgaver til det konkrete system og vores rapport. Hver gang vi skulle planlægge nye sprints, tog vi opgaver fra vores projekt backlog over i det enkelte sprints backlog.
\\\\
Vi har tilføjet reelle eksempler fra projektperioden i bilag af et burndown chart på figur \ref{fig:burndown-chart}, en swim lane med opdelte opgaver fra backlog'en på figur \ref{fig:swim-lane} og vores projekt backlog på figur \ref{fig:projekt-backlog}.
\subsection{Review referater}
I dette afsnit vil der refereres til de vigtigste ting fra sprint reviews i løbet af projektperioden
\subsubsection{Fredag d. 20/11}
Tilstedeværende til mødet: Mikkel, Christoffer, Michael og Peter.
\\\\
Der blev præsenteret hvad der var blevet lavet i løbet af sprintet, som primært var CRUD til håndtering af brugere. Derudover havde vi brugt meget tid på at strukturere systemet, skrive rapport, generel opsætning af teknologier, mm.
Det feedback vi fik var at vi skulle fokusere mere på at lave kernefunktionalitet til systemet, så de funktioner der var vigtige for interessenter skulle vi fokusere på i stedet for f.eks. et brugersystem.
Derudover fik nogle fingerprej om hvilken retning vi skulle gå i med projektet og hvad der var vigtigt for systemet.
\\\\
Nogle punkter vi kunne tage med var,
\begin{itemize}
    \item{Pull eller push af data?}
    \item{Fokus på funktionalitet}
    \item{Vælg en sti og gør den færdig (et diagram ad gangen)}
    \item{Kig evt. på google analytics mht. hentning af data (api keys)}
    \item{Guide til upload/hentning af data}
\end{itemize}
\subsubsection{Fredag d. 4/12}
Tilstedeværende til mødet: Mikkel, Christoffer, Michael og Peter.
\\\\
Præsentation af det daværende sprint. Opgaverne i sprintet indebar hentning og visning af data, upload af API endpoint. Derudover fortalte vi omkring vores research mht. ElasticSearch og hvordan vi havde valgt at vi vil pulle vores data fra API'er frem for at det bliver pushet til vores server. På denne måde kan vi nemmere kontrollere hvad og hvor meget data der er tilgængeligt på vores ElasticSearch server.
Det feedback vi fik var, at vi skulle gøre det mere specifikt hvad der skulle uploades, så det ikke er rå data der bliver uploadet til vores server, som vi skal behandle og lave statistik data ud fra, det skal være ren statistik data der bliver uploadet. På den måde skaber det mere arbejde for dem der skal uploade, men det gør at vi kun har ren statistik tilgængelig på vores server. Det vil ikke give så mange plads problemer, da det ikke fylder specielt meget at have en dato og et tal, i forhold til en masse objekter osv. Det er primært en dato eller et tidspunkt der skal være en parameter, samt en parameter der angiver et tal eller lignende for det givne tidspunkt.
Derudover fik vi lidt feedback på hent og vis data, i forhold til hvordan det skal fungere når der skal laves diagrammer.
Generelt gik det meget bedre end første review og vi var på rette vej i forhold til et mere fuldendt produkt. Generelt skal vi fokusere mere på, at fortsætte i samme retning, men med små-justeringer. På den måde kan vi virkelig få gavn af disse reviews og få noget løbende feedback fra vores interessenter.
\subsubsection{Fredag d. 18/12}
Tilstedeværende til mødet: Mikkel, Michael og Peter.
\\\\
Mikkel præsenterede det daværende sprint, som var Upload af API endpoint, en del af valideringen af API, Gem API data, dynamisk hentning af index/type lister og oprettelse af et diagram.\\
Herudover var der også en demo, hvor der blev gennemgået alle funktioner samt fejlhåndtering i systemet. Grunden til dette, var at vi på dette tidspunkt havde lavet alle de grundlæggende funktioner til at løse
problemstillingen, og vi synes det var tid til at få et overblik.\\
Det feedback vi fik, var at det var rart at se hele flow'et fungere, og at vi kunne få vist noget data i forskellige diagramtyper. (Cirkeldiagram, søjlediagrammer og graf).\\
Dog synes de at vi skal ændre i API strukturen, og den måde det bliver hentet og vist på.
De kunne godt tænke sig at man henter fra et bestemt API, som er bestående af et datasæt der indeholder et timestamp og en værdi. Ved hjælp af denne data skal systemet så kunne
udregne statistik data på årlig, månedlig og daglig basis.\\
Man skal så have mulighed for at vælge enten år, måned eller dag samt et tidsinterval man gerne vil have at ens graf skal repræsentere.\\
Derudover skal det være muligt at lave flere grafer i samme diagram, så det er muligt at sammenligne forskellige datasæt.\\\\
Det var rigtig rart at kunne vise alle funktionerne i programmet samtidig, og få feedback på det samlede system. Interessenterne var derudover generelt glade for produktet, og viste en iver og et
ønske for, at tilføje og ændre i funktionerne til systemet, hvilket bare er fedt at de er så engagerede omkring ens produkt. Det gav noget ekstra blod på tanden til at færdiggøre og skabe et værdifuldt
produkt.
