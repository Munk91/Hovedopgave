\section{Læsevejledning}
Rapporten henvender sig til vejleder, Ordbogen A/S og censor og$ er bygget op i tre større dele:
\begin{itemize}
    \item{En analyserende del}
    \item{En teknisk del}
    \item{En evaluerende del}
\end{itemize}
Hver del indeholder en gennemgående beskrivelse, hvor overskriften er udgangspunktet. 
\\
Opdelingen og rækkefølgen af emnerne, afspejler vores forløb med udviklingen af produktet.
\\\\
En hurtig gennemgang af rapporten kan fåes ved at læse resuméet, delkonklusionen til \hyperlink{delkonklusion-analyse}{analyse} og \hyperlink{delkonklusion-teknik}{teknik}, samt \hyperlink{konklusion}{hovedkonklusionen}.
\\\\
Rapporten er lavet med den tanke at den skal læses i kronologisk rækkefølge, for at få den bedst mulige forståelse af projektet.\\
Først har vi beskrevet hvilke teorier og metoder vi har anvendt, herefter hvordan vi har valgt at implementere produktet,
samt en opsamling på projektet og forløbet som helhed til sidst.

