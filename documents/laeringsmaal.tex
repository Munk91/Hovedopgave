\section{Læringsmål i forbindelse med afsluttende eksamensprojekt}
I projektets forløb vil vi, i samarbejde med Ordbogen A/S, formulere en problemstilling til et konkret problem.
Formålet med hovedopgaven er at dokumentere en IT-løsning der løser virksomhedens problem.
Udover dokumentationen af den konkrete løsning, vil vi dokumentere hvordan vi har analyseret problemer samt
hvilke metoder vi har anvendt til at løse disse, i projektperiodens forløb.
\\\\
Vi vil arbejde som et team bestående af 2 personer; heraf den ene som Scrum master og den anden som Produkt ejer.
Begge personer har lige delt ansvar for udviklingen af produktet og den tilsvarende dokumentation.
\\\\
Under projektperioden vil vores viden og forståelse for at anvende teori i relation til en praksisnær problemstilling blive udfordret.
Vi vil blive udsat for mange problemstillinger der kan opstå på et større IT-projekt, der udvikles i samarbejde med en virksomhed.
\\
Herudover vil vi skulle planlægge, styre og gennemføre projektet med de kendte teknologier vi har tilegnet os fra studiet.
Heriblandt Scrum til planlægning, Git til versionsstyring og metodikker som vi har lært i løbet af studiet til analyse af sekvenser og relationer
i et system.
\\\\
Vi vil i vores rapport gøre rede for hvordan vi har tilpasset forskellige teknikker til vores behov. Derudover vil det kræves at arbejdsprocessen
dokumenteres, i forbindelse med udviklingen af denne.
\\
Dokumentationen vil også indeholde forklaringer om nye teknologier som vi har anvendt til udviklingen af produktet, som f.eks.\ testing, PHP og Javascript
frameworks og ElasticSearch som alternativt værktøj til opbevaring af data.
